\documentclass[11pt]{beamer}

%START To get MATLAB environment
\usepackage[numbered,framed]{matlab-prettifier}
%\usepackage{filecontents}

\let\ph\mlplaceholder % shorter macro
\lstMakeShortInline"

\lstset{
	style              = Matlab-editor,
	basicstyle         = \tiny \ttfamily,
	escapechar         = ",
	mlshowsectionrules = true,
}

%\renewcommand{\lstlistingname}{Algorithm}% Listing -> Algorithm
\renewcommand{\lstlistingname}{Code}% Listing -> Code
%FINISH To get MATLAB environment

%\usepackage{standalone}
%\graphicspath{{../}{../}}
%\graphicspath{{../../figures/}}

\setcounter{tocdepth}{1}

\usepackage[english]{babel}

\usepackage{amsmath}
\usepackage{amsfonts}
\usepackage{amssymb}
\usepackage{graphicx}
\usepackage{tikz}
\usetikzlibrary{shapes.geometric, arrows}
\tikzstyle{startstop} = [rectangle, rounded corners, minimum width=1.7cm, minimum height=0.7cm,text centered, draw=black, fill=red!30]
\tikzstyle{io} = [trapezium, trapezium left angle=70, trapezium right angle=110, minimum width=1.7cm, minimum height=0.7cm, text centered, draw=black, fill=blue!30]
\tikzstyle{process} = [rectangle, minimum width=1.7cm, minimum height=0.7cm, text centered, draw=black, fill=orange!30]
\tikzstyle{decision} = [diamond, minimum width=1.7cm, minimum height=0.7cm, text centered, text width=1.7cm, draw=black, fill=green!30]
\tikzstyle{arrow} = [thick,->,>=stealth]



\setbeamercovered{transparent}
%\usepackage{enumitem}
%\setlist[itemize]{leftmargin=*}

%\usepackage{mhchem}
%\usepackage[utf8]{inputenc}
%\usepackage[T1]{fontenc}
%\numberwithin{equation}{section}



\author[Jose Mendoza-Cortes]{Prof. Jose L. Mendoza-Cortes}
\title[Machine Learning]{Machine Learning}
%\subtitle{Spring '18}
\institute[]
{\scriptsize  
	Scientific Computing Department, Dirac Science Building \\
	Materials Science and Engineering, High Performance Materials Institute\\
	Florida State University\\
	\href{mailto:jmendozacortes@fsu.edu}{jmendozacortes@fsu.edu}\\[3mm]
	
	Condensed Matter Theory, National High Magnetic Field Laboratory\\%[3mm]
	Florida State University\\	
	\href{mailto:mendoza@magnet.fsu.edu}{mendoza@magnet.fsu.edu}\\[3mm]	
	
	Chemical and Biomedical Engineering \\
	Florida State University | Florida A\&M University | College of Engineering \\
	\href{mailto:mendoza@eng.famu.fsu.edu}{mendoza@eng.famu.fsu.edu}\\[3mm]
	Web: \href{http://mendoza.eng.fsu.edu/}{http://mendoza.eng.fsu.edu/}\\%[1mm]
}  

\date{}
\subject{Theory and Computations in Materials, Chemistry and Physics}

%\usetheme{Berkeley}
%\logo{\includegraphics[scale=0.213]{figures/fsu_logo.png}}
%\addtobeamertemplate{navigation symbols}{}{%
%    \usebeamerfont{footline}%
%    \usebeamercolor[fg]{footline}%
%    \hspace{1em}%
%    \insertframenumber/\inserttotalframenumber
%}

\usetheme{Madrid}
%\usecolortheme{beaver}
%\usecolortheme{orchid}

\newif\ifplacelogo % create a new conditional
\placelogotrue % set it to true
\logo{\ifplacelogo
	\includegraphics[width=0.1\linewidth]{figures/fsu_logo.png}
	\includegraphics[width=0.1\linewidth]{figures/famu_logo.png}
	\includegraphics[width=0.1\linewidth]{figures/maglab_logo.png}
	\fi} % replace with your own command

\definecolor{mycustom}{RGB}{0,0,102}       %102,38,38 %128,0,0
%\definecolor{mycustom}{RGB}{128,0,0}       %102,38,38 %128,0,0
%
%\setbeamercolor{structure}{bg=white, fg=custom}
%\setbeamercolor{caption}{fg=custom}

\definecolor{custom}{cmyk}{1,0.5,0,0.47}       %102,38,38 %128,0,0

\setbeamercolor{structure}{bg=white, fg=custom}
\setbeamercolor{caption}{fg=custom}

\setbeamertemplate{navigation symbols}{} %To remove the navigation symbols

%\setbeamercolor{frametitle}{fg=custom}
%\setbeamercolor{framesubtitle}{fg=custom}
\setbeamercolor{titlelike}{parent=structure,bg=gray!20!white}

\setlength\abovecaptionskip{-3pt}
\setbeamertemplate{caption}{%
	\insertcaptionname\,\insertcaptionnumber:\,\insertcaption
}



\usepackage{hyperref}
\hypersetup{colorlinks=true,
	linkcolor=mycustom,
	urlcolor=mycustom}

\abovedisplayskip=0pt
\belowdisplayskip=0pt


\usepackage{pgfpages}
\pgfpagesuselayout{2 on 1}[letterpaper,%landscape,
border shrink=5mm]


	\begin{document}
		
		\placelogotrue % turn the logo off \usetheme{Madrid}
		\maketitle
		
		\placelogofalse % turn the logo off


%*******************************************
%*******************************************
\section{Review on vectorization}
\subsection{Example}

\begin{frame}[fragile]
	\frametitle{\secname: \subsecname}
	\vspace{-2mm}
	\lstinputlisting[caption={Form 1}]{N4_Notes_on_programming_loops/vectorization.m}

	\lstinputlisting[caption={Form 2}]{N4_Notes_on_programming_loops/vectorizationalt.m}	
\end{frame}

%*******************************************
%*******************************************
\section{Review on Anonymous Functions}
%\subsection{Example 1}

\begin{frame}[fragile]
	\frametitle{\secname}
	\vspace{-2mm}
	\lstinputlisting{N4_Notes_on_programming_loops/reviewAnonyFunctions.m}
	\begin{exampleblock}{}
		Notice how we can define two variables in the anonymous function and compare this to the regular way we declare explicit function (where another file is created)
	\end{exampleblock}
\end{frame}

%*******************************************
%*******************************************
\section{Review on `For' Loops}
\subsection{Example 1}

\begin{frame}
	\frametitle{\secname: \subsecname}
	\vspace{-2mm}
	\lstinputlisting{N4_Notes_on_programming_loops/forLoopBasic.m}
	\begin{exampleblock}{}
		The loop can `increase' in negative increments or positive increments. The second example of the code above explore the negative increment. 
	\end{exampleblock}
\end{frame}

%*******************************************
%*******************************************
%\section{Review on Loops}
\subsection{Example 2}

\begin{frame}[fragile]
	\frametitle{\secname: \subsecname}
	\vspace{-2mm}
	\lstinputlisting{N4_Notes_on_programming_loops/Myfactorial.m}
\end{frame}

%*******************************************
%*******************************************
\section{Review on `While' Loops}
\subsection{Typical While}

\begin{frame}[fragile]
	\frametitle{\secname: \subsecname}
	\vspace{-2mm}
	\lstinputlisting{N4_Notes_on_programming_loops/whileLoopUse1.m}
	
	\begin{alertblock}{}
		Notice the way this is written \verb|while condition|. This is the simplest way to do a while loop. Remember that this type of loop will repeat indefinitely if not used with care. 
	\end{alertblock}
\end{frame}

%*******************************************
%*******************************************
%\section{Review on `While' Loops}
\subsection{Alternative While}

\begin{frame}[fragile]
	\frametitle{\secname: \subsecname}
	\vspace{-2mm}
	\lstinputlisting{N4_Notes_on_programming_loops/whileLoopUse2.m}
	\vspace{-2.7mm}	
	\begin{alertblock}{Warning}
	This type of \verb|while(1)| loop will not stop until there is a condition to break it. Thus there must be a \verb|break| command inside the loop. 
	\end{alertblock}	
\end{frame}


%*******************************************
%*******************************************
\section{Using `Pause' command}
%\subsection{Example 1}

\begin{frame}[fragile]
	\frametitle{\secname}
	\vspace{-2mm}
	\lstinputlisting{N4_Notes_on_programming_loops/usingPauseCommand.m}
	\begin{block}{}
		In class we tried different variation of \verb|pause| versus \verb|pause(3)|. Remember that if no number is used in parenthesis, the code will stop indefinitely while if a number is place there, it will wait that number of seconds.
	\end{block}
\end{frame}

%*******************************************
%*******************************************
\section{Using `Animation'}
%\subsection{Example 1}

\begin{frame}[fragile]
	\frametitle{\secname}
	\vspace{-2mm}
	\lstinputlisting[firstline=1, lastline=14]{N4_Notes_on_programming_loops/animationForm2.m}
\end{frame}

\begin{frame}[fragile]
	\frametitle{\secname (continuation)}
	\vspace{-2mm}
	\lstinputlisting[firstline=15, lastline=46]{N4_Notes_on_programming_loops/animationForm2.m}
	\begin{exampleblock}{Exercise}
		In class we tried different variation for animation, can you reproduce the ones that we tried in class?
	\end{exampleblock}
\end{frame}



%*****************
\placelogotrue

\begin{frame}
\frametitle{See you next class}
\vspace{-25pt}

\textbf{\textit{``Just as there is not royal road to geometry, there is no royal road to programming''}}.- Euclid and J. V. Guttag
\vspace{7pt}

\textit{The computer will do what you TELL them to do NOT what you WANT them to do}.- Someone in the Internet (Perhaps)
\vspace{7pt}	

\textit{Think twice, code once}.- Anonymous
\vspace{7pt}

\textit{The sooner you start to code, the longer the program will take}.- R. Carlson\vspace{7pt}

\textit{Any fool can write code that a computer can understand. Good programmers write code that humans can understand}.- M. Fowler
\vspace{7pt}

\textit{Simplicity is the soul of efficiency}.- A. Freeman
\vspace{7pt}

\textit{If you cannot grok the overall structure of a program while taking a shower, you are not ready to code it}.- R. Pattis

\end{frame}

\placelogofalse

\section{Appendix: Scripts included}

%\subsection{}

\begin{frame}
\frametitle{\secname}

\vspace{-7pt}
\lstinputlisting[caption={SumIniFin.m}]{N4_Notes_on_programming_loops/SumIniFin.m}

\vspace{-7pt}
\begin{exampleblock}{}
	Try these commands in your own workstation, i.e. have the lectures on one half side of your screen and Matlab/Octave-GUI on the other half. %This is the best approach to learning this.   
\end{exampleblock}

\begin{alertblock}{}
	Check the scripts/functions under the directory for this note number (X): \newline
	/NX\_Notes\_directory
\end{alertblock}

\end{frame}	


\end{document}